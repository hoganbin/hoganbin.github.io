\documentclass[11pt,twoside]{ctexart}
\usepackage{CJKnumb}
\usepackage{amsmath,amssymb}
\usepackage{calc}
\usepackage{intcalc}
\usepackage{ifthen}
\usepackage{zref-user}
\usepackage{zref-lastpage}
\usepackage{makecell}
\usepackage{interfaces-makecell}
\usepackage{dashrule}
\usepackage{parskip}
\usepackage[paperwidth=195mm,paperheight=270mm,left=30mm,right=25mm,top=20mm,bottom=20mm,includefoot]{geometry}
\usepackage{enumerate}
\usepackage{fancyhdr}
\usepackage{tcolorbox}
\usepackage{graphicx}
\pagestyle{fancy}

%用到的长度变量
\newlength{\wot}
\newlength{\wol}
\newlength{\gmw}
\newlength{\dl}

%长度变量的初始值
\settowidth{\wot}{复核人}
\setlength{\wol}{0.3pt}
\setlength{\gmw}{6em}
\setlength{\dl}{10em}

%页眉设置开始
\renewcommand{\headrulewidth}{0pt}

%装订线开始
\fancyheadoffset[OL,ER]{\gmw}
\fancyhead[OL]{
\ifnum\intcalcMod{\value{page}}{4}=1
\rotatebox{90}
{\begin{minipage}{1.1\textheight}
\begin{center}
省市:\rule[-.2ex]{\dl}{\wol} 学校:\rule[-.2ex]{\dl}{\wol}  姓名:\rule[-.2ex]{\dl}{\wol} 准考证号:\rule[-.2ex]{\dl}{\wol}\\
\tiny \hdashrule[-3ex]{\textheight}{\wol}{3pt}\\[\smallskipamount]
\makebox[0.6\textheight][s]{装订线内不要答题}\\[-3\smallskipamount]
\hdashrule[-3ex]{\textheight}{\wol}{3pt}
\end{center}
\end{minipage} }
\fi
}
\fancyhead[ER]{
\ifnum\intcalcMod{\value{page}}{4}=0
\rotatebox{-90}
{\begin{minipage}{1.1\textheight}
\begin{center}
\tiny \hdashrule[-3ex]{\textheight}{\wol}{3pt}\\[\smallskipamount]
\makebox[0.6\textheight][s]{装订线内不要答题}\\[-3\smallskipamount]
\hdashrule[-3ex]{\textheight}{\wol}{3pt}
\end{center}
\end{minipage} }
\fi
}
%装订线结束
%页眉设置结束

%页脚设置开始
\renewcommand{\footrulewidth}{\wol}
\fancyfoot[C]{\large{{\xingkai 微信公众号之数学的情怀}\qquad 共\zpageref{LastPage}页\quad 第\thepage 页}}
\newcounter{ns}
\newcounter{ts}
\newcounter{nq}
\newcommand{\wns}{\stepcounter{ns}\CJKnumber{\thens}、}
\newcommand{\wq}{\stepcounter{nq}\thenq.}

%大题前计分表格
\newcommand{\tbs}{\begin{tabular}{|c|c|c|}\hline \makebox[\wot]{得分}&\makebox[\wot]{评卷人}&\makebox[\wot]{复核人}\\ \hline
 & &\\ \hline\end{tabular}}

%排版大题前计分表格,序号,题型,大题说明
\newcommand{\ws}[2]{\raisebox{-1ex}{\begin{minipage}[b]{4.6\wot}\tbs\end{minipage}}
\begin{minipage}[t]{\textwidth-6\wot} {\heiti \wns #1 } #2 \end{minipage} }
\makeatletter
\zref@newprop{totalsections}[3]{\arabic{ns}}
\zref@addprop{LastPage}{totalsections}
\AtBeginDocument{
\setcounter{ts}{\zref@extractdefault{LastPage}{totalsections}{3}} }
\makeatother
\linespread{1.618}
\newcommand{\D}{\,\mathrm{d}}
\newcommand{\E}{\mathrm{e}}
\newcommand{\dlim}{\displaystyle \lim }
\newcommand{\dint}{\displaystyle \int }
\newcommand{\sets}[1]{\{ #1 \}}

\setCJKfamilyfont{huawenxingkai}{华文行楷} \newcommand*{\xingkai}{\CJKfamily{huawenxingkai}}%华文行楷
\newcommand{\dis}{\displaystyle}
\newcommand{\Rank}{\mathrm{Rank}\mspace{1mu}}
\newcommand{\Sign}{\mathrm{Sign}\mspace{1mu}}
\newcommand{\rd}{\mspace{1mu}\mathrm{d}}
\newcommand{\diag}{\mathrm{diag}\mspace{1mu}}
\newcommand{\tr}{\mathrm{tr}\mspace{1mu}}
\newcommand{\var}{\mathrm{var}}
\renewcommand{\Re}{\operatorname{Re}}
\renewcommand{\Im}{\operatorname{Im}}
\def\sgn{\mathop{\rm sgn}}
%======================
%试卷头开始
\begin{document}
%试卷标题开始
\begin{center}\vspace{3mm}
      {\xingkai \Large 第十届全国大学生数学竞赛试题}\\[0.8mm]
      { $\left(\text{非数学类, 2018年10月27日}\right)$}\\
\end{center}

%试卷标题结束

%输出"绝密"字样
{\vspace{-1.3mm}\heiti 绝密$\bigstar$启用前}\\[-4\bigskipamount]\\[-12mm]
\begin{center}
\vspace*{2mm}
(16数学$-$胡八一)\\[3mm]
 {考试形式:\underline{~闭卷~}~\hspace{2mm}考试时间:\underline{~~150~~}分钟~\hspace{2mm}满分:~\underline{~~100~~}~分}\\

%根据大题数目自动生成计分总表
\newcounter{tc}
\newcounter{tcsr}
\setcounter{tc}{\value{ts}+3}
\setcounter{tcsr}{\value{tc}-1}
\arrayrulewidth=2\wol 

\vspace*{3.5mm}
\begin{tabular}{|m{3em}<{\centering}|*{11}{m{3.5em}<{\centering}|}}\hline
         题~号 & 一 & 二 & 三  & 四 & 五 &六 &七  &总~~分 \\\hline
		 满~分 & 24 & 8 & 14  & 12 & 14  &14 &14 &100    \\\hline
	 	 得~分 &    &   &     &    &     &   &  &\rule{0pt}{8mm} \\\hline
	\end{tabular}
	\\\vspace*{-1.5mm}
	\begin{equation*}
	\begin{aligned}
	\mbox{注意:}
	&1.\,\mbox{所有答题都须写在试卷密封线右边,写在其他纸上一律无效}.\hspace{12.0cm}\\
	&2.\,\mbox{密封线左边请勿答题,密封线外不得有姓名及相关标记}.\\
	&3.\,\mbox{如答题空白不够,可写在当页背面,并标明题号}.\\[-2mm]
	\end{aligned}
	\end{equation*}	
\end{center}

%试卷头结束

\addvspace{1\bigskipamount}

\ws{\xingkai 填空题}{(\xingkai 本题满分24分,每题6分)\\}\\\\
\wq 设 $\alpha\in\left(0,1\right),\textrm{则}\displaystyle \lim_{n\rightarrow +\infty}\left(\left(n+1\right)^{\alpha}-n^{\alpha}\right)=$\underline{0}.\\

\wq $\textrm{若曲线}y=f\left(x\right)\textrm{是由}\left\{\begin{array}{l}
x=t+\cos t\\
e^y+ty+\sin t=1\\
\end{array}\right.\textrm{确定,则此曲线在}t=0$ 对应点处的\\
切线方程为\underline{$y=1-x$}.\\

\wq $\displaystyle \int{\displaystyle\frac{\ln\left(1+\sqrt{1+x^2}\right)}{\left(1+x^2\right)^{\frac{3}{2}}}}dx=$\underline{$\displaystyle \frac{x}{\sqrt{1+x^2}}\ln\left(x+\sqrt{1+x^2}\right)-\displaystyle \frac{1}{2}\ln\left(1+x^2\right)+C
	$}.\\

\wq $\displaystyle\lim_{x\rightarrow 0}\displaystyle\frac{1-\cos x\sqrt{\cos 2x}\sqrt[3]{\cos 3x}}{x^2}=$\underline{3}.\\
\newpage
\ws {\xingkai 解答题}{(\xingkai 本题满分8分)\\}\\\\
设函数$f(t)$在$t\ne 0$时一阶连续可导,且$f(1)=0$,求函数$f(x^2-y^2)$,使得曲线积分$\displaystyle \int_L{y\left(2-f\left(x^2-y^2\right)\right)}dx+xf\left(x^2-y^2\right)dy
$与路径无关,其中$L$为任一不与直线$y=\pm x$相交的分段光滑闭曲线.
\newpage




\ws {\xingkai 解答题}{(\xingkai 本题满分14分)\\}\\\\
设$f(x)$在区间$[0,1]$上连续,且$1\leqslant f\left(x\right)\leqslant 3$.证明:
\[
0\leqslant\int_0^1{f\left(x\right)dx\int_0^1{\frac{1}{f\left(x\right)}dx\leqslant\frac{4}{3}}}
\]
\newpage

 
 
 
\ws {\xingkai 解答题}{(\xingkai 本题满分12分)\\}\\\\
计算三重积分$\displaystyle
\iiint\limits_{\left(V\right)}{\left(x^2+y^2\right)}dV
$,其中$(V)$是由$x^2+y^2+\left(z-2\right)^2\geqslant 4$,$x^2+y^2+\left(z-1\right)^2\leqslant9$及$z\geqslant 0$所围成的空间图形.
\newpage






\ws {\xingkai 解答题}{(\xingkai 本题满分14分)\\}\\\\
设$f(x,y)$在区域D内可微,且$\sqrt{\left(\displaystyle\frac{\partial f}{\partial x}\right)^2+\left(\displaystyle\frac{\partial f}{\partial y}\right)^2}\leqslant M$,$A\left(x_1,y_1\right),B\left(x_2,y_2\right)$是D内两点,线段AB包含在D内,证明:
\[
|f\left(x_1,y_1\right)-f\left(x_2,y_2\right)|\leqslant M|AB|
\]
其中$|AB|$表示线段$AB$的长度.
\newpage





\ws {\xingkai 解答题}{(\xingkai 本题满分14分)\\}\\\\
证明:对于连续函数$f(x)>0$,有
\[
\ln\int_0^1{f\left(x\right)dx\geqslant\int_0^1{\ln f\left(x\right)dx}}
\]
\newpage




\ws {\xingkai 解答题}{(\xingkai 本题满分14分)\\}\\\\
已知${a_k}$,${b_k}$是正数数列,且$b_{k+1}-b_k\geqslant\delta >0,k=1,2,\cdots $,$\delta$为一切常数,证明:若级数$\displaystyle\sum_{k=1}^{+\infty}{a_k}$收敛,则级数$\displaystyle\sum_{k=1}^{+\infty}{\displaystyle\frac{k\sqrt[k]{\left(a_1a_2\cdots a_k\right)\left(b_1b_2\cdots b_k\right)}}{b_{k+1}b_k}}$收敛.

\mbox{}


%试卷正文结束
\end{document}
